\documentclass [11pt]{article}
%Needed for fancy maths stuff
\usepackage{amsmath}
\usepackage{amssymb}
%Needed to use the H thing
\usepackage{float}

\title{COMS 4030A: Assignment 2}
\author{Tamlin Love\\1438243}
\date{}

\begin{document}
\maketitle
\textbf{Note: }Where applicable, numbers have been rounded to two decimal points. Thus, for example, $0.075$ is rounded to $0.08$.
\section{Question 1: K-Nearest Neighbour Classification}
\subsection{}
\begin{table}[H]
\begin{tabular}{|l|c|l|}
\hline
\textbf{Record \#}      & \multicolumn{1}{l|}{\textbf{Distance}} & \textbf{Label}         \\ \hline
\multicolumn{1}{|c|}{9} & 1.98                                   & \multicolumn{1}{c|}{0} \\ \hline
\textbf{Prediction}     & \multicolumn{2}{c|}{0}                                          \\ \hline
\end{tabular}
\end{table}
\subsection{}
\begin{table}[H]
\begin{tabular}{|c|c|c|}
\hline
\multicolumn{1}{|l|}{\textbf{Record \#}}  & \multicolumn{1}{l|}{\textbf{Distance}} & \multicolumn{1}{l|}{\textbf{Label}} \\ \hline
23                                        & 0.92                                   & 1                                   \\ \hline
28                                        & 1.75                                   & 1                                   \\ \hline
21                                        & 1.82                                   & 1                                   \\ \hline
\multicolumn{1}{|l|}{\textbf{Prediction}} & \multicolumn{2}{c|}{1}                                                       \\ \hline
\end{tabular}
\end{table}
\subsection{}
\begin{table}[H]
\begin{tabular}{|c|c|c|}
\hline
\multicolumn{1}{|l|}{\textbf{Record \#}}  & \multicolumn{1}{l|}{\textbf{Distance}} & \multicolumn{1}{l|}{\textbf{Label}} \\ \hline
32                                        & 0.81                                   & 1                                   \\ \hline
39                                        & 0.95                                   & 1                                   \\ \hline
21                                        & 1.00                                   & 1                                   \\ \hline
28                                        & 1.22                                   & 1                                   \\ \hline
23                                        & 2.20                                   & 1                                   \\ \hline
\multicolumn{1}{|l|}{\textbf{Prediction}} & \multicolumn{2}{c|}{1}                                                       \\ \hline
\end{tabular}
\end{table}
\subsection{}
\begin{table}[H]
\begin{tabular}{|l|c|c|}
\hline
\textbf{K-NN} & \multicolumn{1}{l|}{\textbf{Training Error Rate}} & \multicolumn{1}{l|}{\textbf{Test Error Rate}} \\ \hline
1-NN          & 0.00                                              & 0.05                                          \\ \hline
3-NN          & 0.00                                              & 0.08                                          \\ \hline
5-NN          & 0.00                                              & 0.05                                          \\ \hline
7-NN          & 0.00                                              & 0.05                                          \\ \hline
9-NN          & 0.05                                              & 0.08                                          \\ \hline
11-NN         & 0.05                                              & 0.08                                          \\ \hline
13-NN         & 0.05                                              & 0.03                                          \\ \hline
\end{tabular}
\end{table}
\section{Question 2: Logistic Regression}
\subsection{}
The range of $h_{\Theta}(x)$ is the interval $[0,1]$.
\\
The classifier cannot learn from an arbitrary data set if $\beta = 0$, as this causes $h_{\Theta}(x)=0.5$ for all values of $x$.
\subsection{}
We begin with the following definition for the cost function
\begin{equation}\label{eq:costFunction}
J(\Theta) = \frac{1}{4N} \sum_{n=1}^{N}(h_{\Theta}(x^{(n)})-y^{(n)})^4
\end{equation}
and the update rule
\begin{equation}\label{eq:updateRule}
\theta_{k} \gets \theta_{k} - \alpha \frac{\partial J(\Theta)}{\partial \theta_{k}}
\end{equation}
and the activation function
\begin{equation}\label{eq:activationFunction}
h_{\Theta}(x^{(n)}) = \frac{1}{1+e^{-\beta\Theta^{T}x^{(n)}}}
\end{equation}
Now, by the chain rule, we have that
\begin{equation}\label{eq:chainRule}
\frac{\partial J(\Theta)}{\partial \theta_{k}} = \frac{\partial J(\Theta)}{\partial h_{\Theta}(x^{(n)})} \frac{\partial h_{\Theta}(x^{(n)})}{\partial \theta_{k}}
\end{equation}
The partial derivative of equation \ref{eq:costFunction} with respect to $h_{\Theta}(x^{(n)})$ is
\begin{equation}\label{eq:dJdh}
\frac{\partial J(\Theta)}{\partial h_{\Theta}(x^{(n)})} = \frac{1}{N} \sum_{n=1}^{N}(h_{\Theta}(x^{(n)})-y^{(n)})^3
\end{equation}
The partial derivative of equation \ref{eq:activationFunction} with respect to $\theta_{k}$ is
\begin{equation}\label{eq:dhdth}
\begin{split}
\frac{\partial h_{\Theta}(x^{(n)})}{\partial \theta_{k}} & = \frac{\partial}{\partial \theta_{k}}\Bigg(\frac{1}{1+e^{-\beta\Theta^{T}x^{(n)}}}\Bigg) \\
& = \frac{\partial}{\partial \theta_{k}}\Bigg(\frac{1}{1+\prod_{i=0}^{d}e^{-\beta \theta_{i} x_{i}^{(n)}}}\Bigg) \\
& = \frac{e^{-\Theta^{T}x^{(n)}}}{1+e^{-\Theta^{T}x^{(n)}}}\beta x_{k}^{(n)} \\
& = \beta h_{\Theta}(x^{(n)})(1-h_{\Theta}(x^{(n)}))x_{k}^{(n)}
\end{split}
\end{equation}
Substituting equations \ref{eq:dJdh} and \ref{eq:dhdth} back into equation \ref{eq:chainRule} yields
\begin{equation}\label{eq:dJdth}
\frac{\partial J(\Theta)}{\partial \theta_{k}} = \frac{\beta}{N}\sum_{n=1}^{N}(h_{\Theta}(x^{(n)})-y^{(n)})^{3}h_{\Theta}(x^{(n)})(1-h_{\Theta}(x^{(n)}))x_{k}^{(n)}
\end{equation}
Finally, we substitute equation \ref{eq:dJdth} into equation \ref{eq:updateRule} to obtain the closed form update rule
\begin{equation*}
\theta_{k} \gets \theta_{k} - \frac{\alpha \beta}{N}\sum_{n=1}^{N}(h_{\Theta}(x^{(n)})-y^{(n)})^{3}h_{\Theta}(x^{(n)})(1-h_{\Theta}(x^{(n)}))x_{k}^{(n)}
\end{equation*}
\subsection{}
\begin{table}[H]
\begin{tabular}{|l|c|c|}
\hline
\textbf{Testing Record \#} & \multicolumn{1}{l|}{\textbf{Classifier Output $h_{\Theta}(x^{(n)})$}} & \multicolumn{1}{l|}{\textbf{Final Output}} \\ \hline
5                          & 0.01                                            & 0                                          \\ \hline
10                         & 0.06                                            & 0                                          \\ \hline
15                         & 0.13                                            & 0                                          \\ \hline
20                         & 0.09                                            & 0                                          \\ \hline
25                         & 0.63                                            & 1                                          \\ \hline
30                         & 0.87                                            & 1                                          \\ \hline
35                         & 0.58                                            & 1                                          \\ \hline
40                         & 0.57                                            & 1                                          \\ \hline
\end{tabular}
\end{table}
\section{Question 3: Artificial Neural Networks}
\subsection{}
\begin{equation}\label{eq:ANN_h}
\begin{split}
h_{\Theta}(x) & = g_{3}(g_{2}(g_{1}(\sum_{i=0}^{4}\theta_{i}x_{i})\theta_{6}+\theta_{5}x_{0})\theta_{7}+\theta_{8}x_{0}) \\
& = \frac{\theta_{6}\theta_{7}}{1+e^{-\sum_{i=0}^{4}\theta_{i}x_{i}}} + (\theta_{5}\theta_{7}+\theta_{8})x_{0}
\end{split}
\end{equation}
\subsection{}
We define the cost function to be
\begin{equation}\label{eq:ANN_costFunction}
J(\Theta) = \frac{1}{2N}\sum_{n=1}^{N}(h_{\Theta}(x^{(n)})-y^{(n)})^2
\end{equation}
we also note that the update rule remains the same as in equation \ref{eq:updateRule}, and that the value of $h_{\Theta}(x^{(n)})$ is the same as in equation \ref{eq:ANN_h}.
\\
Using the chain rule in equation \ref{eq:chainRule}, and substituting the appropriate values for $J(\Theta)$ and $h_{\Theta}(x^{(n)})$, we have the partial derivative with respect to $h_{\Theta}(x^{(n)})$
\begin{equation}\label{eq:ANN_dJdh}
\frac{\partial J(\Theta)}{\partial h_{\Theta}(x^{(n)})} = \frac{1}{N}\sum_{n=1}^{N}(h_{\Theta}(x^{(n)})-y^{(n)})
\end{equation}
Now we take the partial derivative with respect to $\theta_{k}$ for $k = 5,6,7,8$
\begin{equation}\label{eq:ANN_dhdth}
\begin{split}
\frac{\partial h_{\Theta}(x^{(n)})}{\partial \theta_{5}} & = \theta_{7}x_{0}^{(n)} \\
\frac{\partial h_{\Theta}(x^{(n)})}{\partial \theta_{6}} & = \theta_{7}g_{1}(\sum_{i=0}^{4}\theta_{i}x_{i}^{(n)}) = \frac{\theta_{7}}{1+e^{-\sum_{i=0}^{4}\theta_{i}x_{i}^{(n)}}} \\
\frac{\partial h_{\Theta}(x^{(n)})}{\partial \theta_{7}} & = \theta_{5}x_{0}^{(n)}+\theta_{6}g_{1}(\sum_{i=0}^{4}\theta_{i}x_{i}^{(n)}) \\
& = \theta_{5}x_{0}^{(n)} + \frac{\theta_{6}}{1+e^{-\sum_{i=0}^{4}\theta_{i}x_{i}^{(n)}}} \\
\frac{\partial h_{\Theta}(x^{(n)})}{\partial \theta_{8}} & = x_{0}^{(n)}
\end{split}
\end{equation}
Substituting equations \ref{eq:ANN_dJdh} and \ref{eq:ANN_dhdth} into equation \ref{eq:chainRule}, we arrive at
\begin{equation*}
\begin{split}
\frac{\partial J(\Theta)}{\partial \theta_{5}} & = \frac{1}{N}\sum_{n=1}^{N}(h_{\Theta}(x^{(n)})-y^{(n)}) \theta_{7}x_{0}^{(n)} \\
\frac{\partial J(\Theta)}{\partial \theta_{6}} & = \frac{1}{N}\sum_{n=1}^{N}(h_{\Theta}(x^{(n)})-y^{(n)})\theta_{7}g_{1}(\sum_{i=0}^{4}\theta_{i}x_{i}^{(n)}) \\
\frac{\partial J(\Theta)}{\partial \theta_{7}} & = \frac{1}{N}\sum_{n=1}^{N}(h_{\Theta}(x^{(n)})-y^{(n)})(\theta_{5}x_{0}^{(n)}+\theta_{6}g_{1}(\sum_{i=0}^{4}\theta_{i}x_{i}^{(n)})) \\
\frac{\partial J(\Theta)}{\partial \theta_{8}} & = \frac{1}{N}\sum_{n=1}^{N}(h_{\Theta}(x^{(n)})-y^{(n)})x_{0}
\end{split}
\end{equation*}
which, when substituted into equation \ref{eq:updateRule}, yields the closed form update rules for the selected $\theta_{i}$
\begin{equation*}
\begin{split}
\theta_{5} & \gets \theta_{5} - \alpha \frac{1}{N}\sum_{n=1}^{N}(h_{\Theta}(x^{(n)})-y^{(n)}) \theta_{7}x_{0}^{(n)} \\
\theta_{6} & \gets \theta_{6} - \alpha \frac{1}{N}\sum_{n=1}^{N}(h_{\Theta}(x^{(n)})-y^{(n)})\theta_{7}g_{1}(\sum_{i=0}^{4}\theta_{i}x_{i}^{(n)}) \\
\theta_{7} & \gets \theta_{7} - \alpha \frac{1}{N}\sum_{n=1}^{N}(h_{\Theta}(x^{(n)})-y^{(n)})(\theta_{5}x_{0}^{(n)}+\theta_{6}g_{1}(\sum_{i=0}^{4}\theta_{i}x_{i}^{(n)})) \\
\theta_{8} & \gets \theta_{8} - \alpha \frac{1}{N}\sum_{n=1}^{N}(h_{\Theta}(x^{(n)})-y^{(n)})x_{0}
\end{split}
\end{equation*}
\end{document}